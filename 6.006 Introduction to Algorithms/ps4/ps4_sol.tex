%
% 6.006 problem set 4
%
\documentclass[12pt,twoside]{article}
\setlength{\oddsidemargin}{0pt}
\setlength{\evensidemargin}{0pt}
\setlength{\textwidth}{6.5in}
\setlength{\topmargin}{0in}
\setlength{\textheight}{8.5in}
\begin{document}
Problem 4-1.

  a: 3

  b: 4

  c: 5 Correct:7

  d: In practical, m gradually increasing, but k is a constant. We don't know how to 
  choose k. If k is small, when m becomes big and increase fast, a k-size bigger 
  resize will be useless and frequent. If k is big, when m is small and seldom 
  increases, $k-m$ memory will be wasted. In theoretical, if we insert n elements,
  the total cost will be $O(1+k+2k+\ddots+n) = O(n^2)$ and the amortized cost will
  be $O(n)$, which is bad.

  Addition:the computer will play more nicely with operations based around
  doubling (doubling is a fast operation, allocating memory blocks of sizes 
  that are powers of two has plenty of advantages, etc)\\

Problem 4-2.
  a: 1

  b: 2 Correct:3




\end{document}